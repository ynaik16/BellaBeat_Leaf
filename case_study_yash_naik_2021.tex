% Options for packages loaded elsewhere
\PassOptionsToPackage{unicode}{hyperref}
\PassOptionsToPackage{hyphens}{url}
%
\documentclass[
]{article}
\usepackage{amsmath,amssymb}
\usepackage{lmodern}
\usepackage{ifxetex,ifluatex}
\ifnum 0\ifxetex 1\fi\ifluatex 1\fi=0 % if pdftex
  \usepackage[T1]{fontenc}
  \usepackage[utf8]{inputenc}
  \usepackage{textcomp} % provide euro and other symbols
\else % if luatex or xetex
  \usepackage{unicode-math}
  \defaultfontfeatures{Scale=MatchLowercase}
  \defaultfontfeatures[\rmfamily]{Ligatures=TeX,Scale=1}
\fi
% Use upquote if available, for straight quotes in verbatim environments
\IfFileExists{upquote.sty}{\usepackage{upquote}}{}
\IfFileExists{microtype.sty}{% use microtype if available
  \usepackage[]{microtype}
  \UseMicrotypeSet[protrusion]{basicmath} % disable protrusion for tt fonts
}{}
\makeatletter
\@ifundefined{KOMAClassName}{% if non-KOMA class
  \IfFileExists{parskip.sty}{%
    \usepackage{parskip}
  }{% else
    \setlength{\parindent}{0pt}
    \setlength{\parskip}{6pt plus 2pt minus 1pt}}
}{% if KOMA class
  \KOMAoptions{parskip=half}}
\makeatother
\usepackage{xcolor}
\IfFileExists{xurl.sty}{\usepackage{xurl}}{} % add URL line breaks if available
\IfFileExists{bookmark.sty}{\usepackage{bookmark}}{\usepackage{hyperref}}
\hypersetup{
  pdftitle={BellaBeat Leaf Case Study},
  pdfauthor={Yash Naik},
  hidelinks,
  pdfcreator={LaTeX via pandoc}}
\urlstyle{same} % disable monospaced font for URLs
\usepackage[margin=1in]{geometry}
\usepackage{color}
\usepackage{fancyvrb}
\newcommand{\VerbBar}{|}
\newcommand{\VERB}{\Verb[commandchars=\\\{\}]}
\DefineVerbatimEnvironment{Highlighting}{Verbatim}{commandchars=\\\{\}}
% Add ',fontsize=\small' for more characters per line
\usepackage{framed}
\definecolor{shadecolor}{RGB}{248,248,248}
\newenvironment{Shaded}{\begin{snugshade}}{\end{snugshade}}
\newcommand{\AlertTok}[1]{\textcolor[rgb]{0.94,0.16,0.16}{#1}}
\newcommand{\AnnotationTok}[1]{\textcolor[rgb]{0.56,0.35,0.01}{\textbf{\textit{#1}}}}
\newcommand{\AttributeTok}[1]{\textcolor[rgb]{0.77,0.63,0.00}{#1}}
\newcommand{\BaseNTok}[1]{\textcolor[rgb]{0.00,0.00,0.81}{#1}}
\newcommand{\BuiltInTok}[1]{#1}
\newcommand{\CharTok}[1]{\textcolor[rgb]{0.31,0.60,0.02}{#1}}
\newcommand{\CommentTok}[1]{\textcolor[rgb]{0.56,0.35,0.01}{\textit{#1}}}
\newcommand{\CommentVarTok}[1]{\textcolor[rgb]{0.56,0.35,0.01}{\textbf{\textit{#1}}}}
\newcommand{\ConstantTok}[1]{\textcolor[rgb]{0.00,0.00,0.00}{#1}}
\newcommand{\ControlFlowTok}[1]{\textcolor[rgb]{0.13,0.29,0.53}{\textbf{#1}}}
\newcommand{\DataTypeTok}[1]{\textcolor[rgb]{0.13,0.29,0.53}{#1}}
\newcommand{\DecValTok}[1]{\textcolor[rgb]{0.00,0.00,0.81}{#1}}
\newcommand{\DocumentationTok}[1]{\textcolor[rgb]{0.56,0.35,0.01}{\textbf{\textit{#1}}}}
\newcommand{\ErrorTok}[1]{\textcolor[rgb]{0.64,0.00,0.00}{\textbf{#1}}}
\newcommand{\ExtensionTok}[1]{#1}
\newcommand{\FloatTok}[1]{\textcolor[rgb]{0.00,0.00,0.81}{#1}}
\newcommand{\FunctionTok}[1]{\textcolor[rgb]{0.00,0.00,0.00}{#1}}
\newcommand{\ImportTok}[1]{#1}
\newcommand{\InformationTok}[1]{\textcolor[rgb]{0.56,0.35,0.01}{\textbf{\textit{#1}}}}
\newcommand{\KeywordTok}[1]{\textcolor[rgb]{0.13,0.29,0.53}{\textbf{#1}}}
\newcommand{\NormalTok}[1]{#1}
\newcommand{\OperatorTok}[1]{\textcolor[rgb]{0.81,0.36,0.00}{\textbf{#1}}}
\newcommand{\OtherTok}[1]{\textcolor[rgb]{0.56,0.35,0.01}{#1}}
\newcommand{\PreprocessorTok}[1]{\textcolor[rgb]{0.56,0.35,0.01}{\textit{#1}}}
\newcommand{\RegionMarkerTok}[1]{#1}
\newcommand{\SpecialCharTok}[1]{\textcolor[rgb]{0.00,0.00,0.00}{#1}}
\newcommand{\SpecialStringTok}[1]{\textcolor[rgb]{0.31,0.60,0.02}{#1}}
\newcommand{\StringTok}[1]{\textcolor[rgb]{0.31,0.60,0.02}{#1}}
\newcommand{\VariableTok}[1]{\textcolor[rgb]{0.00,0.00,0.00}{#1}}
\newcommand{\VerbatimStringTok}[1]{\textcolor[rgb]{0.31,0.60,0.02}{#1}}
\newcommand{\WarningTok}[1]{\textcolor[rgb]{0.56,0.35,0.01}{\textbf{\textit{#1}}}}
\usepackage{graphicx}
\makeatletter
\def\maxwidth{\ifdim\Gin@nat@width>\linewidth\linewidth\else\Gin@nat@width\fi}
\def\maxheight{\ifdim\Gin@nat@height>\textheight\textheight\else\Gin@nat@height\fi}
\makeatother
% Scale images if necessary, so that they will not overflow the page
% margins by default, and it is still possible to overwrite the defaults
% using explicit options in \includegraphics[width, height, ...]{}
\setkeys{Gin}{width=\maxwidth,height=\maxheight,keepaspectratio}
% Set default figure placement to htbp
\makeatletter
\def\fps@figure{htbp}
\makeatother
\setlength{\emergencystretch}{3em} % prevent overfull lines
\providecommand{\tightlist}{%
  \setlength{\itemsep}{0pt}\setlength{\parskip}{0pt}}
\setcounter{secnumdepth}{-\maxdimen} % remove section numbering
\ifluatex
  \usepackage{selnolig}  % disable illegal ligatures
\fi

\title{BellaBeat Leaf Case Study}
\author{Yash Naik}
\date{8/18/2021}

\begin{document}
\maketitle

\hypertarget{bellabeat-product-analysis-leaf-wearable}{%
\section{BellaBeat Product Analysis : Leaf
wearable}\label{bellabeat-product-analysis-leaf-wearable}}

\hypertarget{deliverables}{%
\subsection{Deliverables}\label{deliverables}}

\begin{itemize}
\tightlist
\item
  A clear summary of the business task
\item
  A description of all data sources used
\item
  Documentation of any cleaning or manipulation of data
\item
  A summary of your analysis
\item
  Supporting visualizations and key findings
\item
  Your top high-level content recommendations based on your analysis
\end{itemize}

\hypertarget{business-task}{%
\subsection{Business Task}\label{business-task}}

In this project I will analyze smart device usage data in order to gain
insight into how consumers use non-Bellabeat smart devices. I will then
select one Bellabeat product to apply these insights to, in my
presentation.

\hypertarget{data-sources}{%
\subsection{Data Sources}\label{data-sources}}

Data source provided by the company

\begin{itemize}
\tightlist
\item
  Data from Mobius Kaggle file. Access data file
  \href{https://www.kaggle.com/arashnic/fitbit}{here} Datasets contains
  observations of 30 people from 2016-03-12 to 2016-05-12.
\end{itemize}

\begin{verbatim}
## -- Attaching packages --------------------------------------- tidyverse 1.3.1 --
\end{verbatim}

\begin{verbatim}
## v ggplot2 3.3.5     v purrr   0.3.4
## v tibble  3.1.3     v dplyr   1.0.7
## v tidyr   1.1.3     v stringr 1.4.0
## v readr   2.0.1     v forcats 0.5.1
\end{verbatim}

\begin{verbatim}
## -- Conflicts ------------------------------------------ tidyverse_conflicts() --
## x dplyr::filter() masks stats::filter()
## x dplyr::lag()    masks stats::lag()
\end{verbatim}

\hypertarget{downloaded-datasets}{%
\subsection{Downloaded Datasets}\label{downloaded-datasets}}

Leaf wearable devices is women-centric device primarily designed to
track their activities, sleep schedules, hydration levels, hygene, etc.
I will be analyzing Fitbit smart wearable device which provides tracking
of some of the similar activities if not all. So, I have included these
below mentioned datasets for my analysis.

\begin{itemize}
\tightlist
\item
  dailyActivity\_merged
\item
  dailySteps\_merged
\item
  minuteSleep\_merged
\item
  sleepDay\_merged
\end{itemize}

\begin{verbatim}
## Rows: 413 Columns: 5
\end{verbatim}

\begin{verbatim}
## -- Column specification --------------------------------------------------------
## Delimiter: ","
## chr (1): SleepDay
## dbl (4): Id, TotalSleepRecords, TotalMinutesAsleep, TotalTimeInBed
\end{verbatim}

\begin{verbatim}
## 
## i Use `spec()` to retrieve the full column specification for this data.
## i Specify the column types or set `show_col_types = FALSE` to quiet this message.
\end{verbatim}

\begin{verbatim}
## Rows: 940 Columns: 15
\end{verbatim}

\begin{verbatim}
## -- Column specification --------------------------------------------------------
## Delimiter: ","
## chr  (1): ActivityDate
## dbl (14): Id, TotalSteps, TotalDistance, TrackerDistance, LoggedActivitiesDi...
\end{verbatim}

\begin{verbatim}
## 
## i Use `spec()` to retrieve the full column specification for this data.
## i Specify the column types or set `show_col_types = FALSE` to quiet this message.
\end{verbatim}

\begin{verbatim}
## Rows: 188521 Columns: 4
\end{verbatim}

\begin{verbatim}
## -- Column specification --------------------------------------------------------
## Delimiter: ","
## chr (1): date
## dbl (3): Id, value, logId
\end{verbatim}

\begin{verbatim}
## 
## i Use `spec()` to retrieve the full column specification for this data.
## i Specify the column types or set `show_col_types = FALSE` to quiet this message.
\end{verbatim}

\begin{verbatim}
## Rows: 940 Columns: 3
\end{verbatim}

\begin{verbatim}
## -- Column specification --------------------------------------------------------
## Delimiter: ","
## chr (1): ActivityDay
## dbl (2): Id, StepTotal
\end{verbatim}

\begin{verbatim}
## 
## i Use `spec()` to retrieve the full column specification for this data.
## i Specify the column types or set `show_col_types = FALSE` to quiet this message.
\end{verbatim}

\hypertarget{explore-the-datasets}{%
\paragraph{Explore the datasets}\label{explore-the-datasets}}

\begin{verbatim}
## # A tibble: 6 x 5
##           Id SleepDay              TotalSleepRecor~ TotalMinutesAsl~ TotalTimeInBed
##        <dbl> <chr>                            <dbl>            <dbl>          <dbl>
## 1 1503960366 4/12/2016 12:00:00 AM                1              327            346
## 2 1503960366 4/13/2016 12:00:00 AM                2              384            407
## 3 1503960366 4/15/2016 12:00:00 AM                1              412            442
## 4 1503960366 4/16/2016 12:00:00 AM                2              340            367
## 5 1503960366 4/17/2016 12:00:00 AM                1              700            712
## 6 1503960366 4/19/2016 12:00:00 AM                1              304            320
\end{verbatim}

\begin{verbatim}
## # A tibble: 6 x 15
##           Id ActivityDate TotalSteps TotalDistance TrackerDistance LoggedActivitie~
##        <dbl> <chr>             <dbl>         <dbl>           <dbl>            <dbl>
## 1 1503960366 4/12/2016         13162          8.5             8.5                 0
## 2 1503960366 4/13/2016         10735          6.97            6.97                0
## 3 1503960366 4/14/2016         10460          6.74            6.74                0
## 4 1503960366 4/15/2016          9762          6.28            6.28                0
## 5 1503960366 4/16/2016         12669          8.16            8.16                0
## 6 1503960366 4/17/2016          9705          6.48            6.48                0
## # ... with 9 more variables: VeryActiveDistance <dbl>,
## #   ModeratelyActiveDistance <dbl>, LightActiveDistance <dbl>,
## #   SedentaryActiveDistance <dbl>, VeryActiveMinutes <dbl>,
## #   FairlyActiveMinutes <dbl>, LightlyActiveMinutes <dbl>,
## #   SedentaryMinutes <dbl>, Calories <dbl>
\end{verbatim}

\begin{verbatim}
## # A tibble: 6 x 3
##           Id ActivityDay StepTotal
##        <dbl> <chr>           <dbl>
## 1 1503960366 4/12/2016       13162
## 2 1503960366 4/13/2016       10735
## 3 1503960366 4/14/2016       10460
## 4 1503960366 4/15/2016        9762
## 5 1503960366 4/16/2016       12669
## 6 1503960366 4/17/2016        9705
\end{verbatim}

\begin{verbatim}
## # A tibble: 6 x 4
##           Id date                 value       logId
##        <dbl> <chr>                <dbl>       <dbl>
## 1 1503960366 4/12/2016 2:47:30 AM     3 11380564589
## 2 1503960366 4/12/2016 2:48:30 AM     2 11380564589
## 3 1503960366 4/12/2016 2:49:30 AM     1 11380564589
## 4 1503960366 4/12/2016 2:50:30 AM     1 11380564589
## 5 1503960366 4/12/2016 2:51:30 AM     1 11380564589
## 6 1503960366 4/12/2016 2:52:30 AM     1 11380564589
\end{verbatim}

Checking the maximum and minimum distance values in activity dataset to
look for potential outliers

\begin{verbatim}
## [1] 28.03
\end{verbatim}

\begin{verbatim}
## [1] 0
\end{verbatim}

\begin{verbatim}
## [1] 796
\end{verbatim}

\begin{verbatim}
## [1] 58
\end{verbatim}

We can observe that minimum value for TrackerDistance is 0 which is an
indication to examine further.

\begin{verbatim}
## # A tibble: 2 x 2
##   `activity_log$TrackerDistance == 0`     n
##   <lgl>                               <int>
## 1 FALSE                                 862
## 2 TRUE                                   78
\end{verbatim}

We can see that out of 940 results 78 are 0 which accounts for roughly
5\% missing values from daily activities, so it will be good idea to
filter out these missing values.

\begin{verbatim}
## # A tibble: 6 x 15
##           Id ActivityDate TotalSteps TotalDistance TrackerDistance LoggedActivitie~
##        <dbl> <chr>             <dbl>         <dbl>           <dbl>            <dbl>
## 1 1503960366 4/12/2016         13162          8.5             8.5                 0
## 2 1503960366 4/13/2016         10735          6.97            6.97                0
## 3 1503960366 4/14/2016         10460          6.74            6.74                0
## 4 1503960366 4/15/2016          9762          6.28            6.28                0
## 5 1503960366 4/16/2016         12669          8.16            8.16                0
## 6 1503960366 4/17/2016          9705          6.48            6.48                0
## # ... with 9 more variables: VeryActiveDistance <dbl>,
## #   ModeratelyActiveDistance <dbl>, LightActiveDistance <dbl>,
## #   SedentaryActiveDistance <dbl>, VeryActiveMinutes <dbl>,
## #   FairlyActiveMinutes <dbl>, LightlyActiveMinutes <dbl>,
## #   SedentaryMinutes <dbl>, Calories <dbl>
\end{verbatim}

Next, we will calculate summary statistics of the filtered activity data
and create different levels for activity.

\begin{verbatim}
## # A tibble: 6 x 5
##           Id total_dist avg_dist num_dist distance_level   
##        <dbl>      <dbl>    <dbl>    <int> <chr>            
## 1 1503960366      242.      8.07       30 Highly Active    
## 2 1624580081      121.      3.91       31 lowly Actively   
## 3 1644430081      159.      5.30       30 Moderately Active
## 4 1844505072       52.9     2.64       20 lowly Actively   
## 5 1927972279       19.7     1.16       17 lowly Actively   
## 6 2022484408      251.      8.08       31 Highly Active
\end{verbatim}

Creating a scatter plot to verify if there is any positive correlation
between tracking distance and average distance.

\includegraphics{case_study_yash_naik_2021_files/figure-latex/plot1-1.pdf}
From the plot above, we can observe a \emph{little} \emph{positive}
\emph{correlation} between tracking entries and average distance
covered. All the users who are highly active have logged their entries
more than 20 times. Similarly, moderately active users, have
comparatively more tracking entries than lowly\_active users.

Now, we can extend our analysis to dig deeper and check if users who are
highly active, have slept for longer.

\begin{verbatim}
## # A tibble: 6 x 5
##           Id total_sleep_min average_sleep_min num_sleep_entries sleep_levels   
##        <dbl>           <dbl>             <dbl>             <int> <chr>          
## 1 1503960366            9007              360.                25 Moderately Res~
## 2 1644430081            1176              294                  4 Poorly Rested  
## 3 1844505072            1956              652                  3 Well Rested    
## 4 1927972279            2085              417                  5 Moderately Res~
## 5 2026352035           14173              506.                28 Well Rested    
## 6 2320127002              61               61                  1 Not Rested
\end{verbatim}

Now, we will again create a scatter plot to check for any correlation
between average sleep and number of sleep entries.

\begin{verbatim}
## `geom_smooth()` using method = 'loess' and formula 'y ~ x'
\end{verbatim}

\includegraphics{case_study_yash_naik_2021_files/figure-latex/sleep_plot-1.pdf}
Yes, we can observe a \emph{positive} \emph{correlation} between users
who have more sleep entries and the number of hours they have slept.

\hypertarget{to-analyse-a-pattern-between-activity-and-sleep-we-can-combine-the-two-summary-dataframes.}{%
\subsubsection{To analyse a pattern between activity and sleep, we can
combine the two summary
dataframes.}\label{to-analyse-a-pattern-between-activity-and-sleep-we-can-combine-the-two-summary-dataframes.}}

Combining \emph{summary\_activity} and \emph{summary\_sleep} to create a
single dataframe.

\begin{verbatim}
##           Id total_dist avg_dist num_dist    distance_level total_sleep_min
## 1 1503960366     242.10 8.070000       30     Highly Active            9007
## 2 1644430081     158.86 5.295333       30 Moderately Active            1176
## 3 1844505072      52.89 2.644500       20    lowly Actively            1956
## 4 1927972279      19.67 1.157059       17    lowly Actively            2085
## 5 2026352035     107.10 3.454839       31    lowly Actively           14173
## 6 2320127002      98.82 3.187742       31    lowly Actively              61
##   average_sleep_min num_sleep_entries      sleep_levels
## 1          360.2800                25 Moderately Rested
## 2          294.0000                 4     Poorly Rested
## 3          652.0000                 3       Well Rested
## 4          417.0000                 5 Moderately Rested
## 5          506.1786                28       Well Rested
## 6           61.0000                 1        Not Rested
\end{verbatim}

We will again visualize to check if there is any correlation between
users who are highly active and who have slept for longer duration.

\hypertarget{visualization}{%
\subsection{Visualization}\label{visualization}}

\includegraphics{case_study_yash_naik_2021_files/figure-latex/plot2-1.pdf}
\includegraphics{case_study_yash_naik_2021_files/figure-latex/plot2-2.pdf}

\begin{verbatim}
## `geom_smooth()` using method = 'loess' and formula 'y ~ x'
\end{verbatim}

\includegraphics{case_study_yash_naik_2021_files/figure-latex/plot2-3.pdf}

\begin{verbatim}
## `geom_smooth()` using method = 'loess' and formula 'y ~ x'
\end{verbatim}

\includegraphics{case_study_yash_naik_2021_files/figure-latex/plot2-4.pdf}
Initially, I thought there would be a strong positive correlation
between average distance tracking and sleeping i.e.~A person who is very
actively should have rested for a longer duration than someone who is
less active on an average. Looking at these trends, I am surprised that
my assumption has been refuted as after visualizing these trends, it
appears that there is \emph{little} to \emph{no} correlation between the
two things.

\begin{Shaded}
\begin{Highlighting}[]
\FunctionTok{ggplot}\NormalTok{(}\AttributeTok{data =}\NormalTok{ activity\_sleep) }\SpecialCharTok{+} \FunctionTok{geom\_line}\NormalTok{(}\FunctionTok{aes}\NormalTok{(}\AttributeTok{x =}\NormalTok{ average\_sleep\_min, }\AttributeTok{y =}\NormalTok{ avg\_dist))}\SpecialCharTok{+}
  \FunctionTok{labs}\NormalTok{(}\AttributeTok{title =} \StringTok{"Average Distance tracked VS Sleep levels"}\NormalTok{)}\SpecialCharTok{+}
  \FunctionTok{facet\_wrap}\NormalTok{(}\SpecialCharTok{\textasciitilde{}}\NormalTok{distance\_level)}
\end{Highlighting}
\end{Shaded}

\includegraphics{case_study_yash_naik_2021_files/figure-latex/unnamed-chunk-1-1.pdf}
* Thus, we can observe that a majority of people who tracked as average
distance of \textgreater{} 7 Km rested for less than 400 minutes (6
hours, 30 mins)

\begin{itemize}
\tightlist
\item
  On Contrary, people who were moderately active, slept for longer than
  400 minutes.
\end{itemize}

\hypertarget{summary} of the data
  from the sleep log was \emph{missing}, which could mean different
  things like - some users did not wear the device while sleeping or the
  device got discharged at night or there was some technical issue in
  the device which resulted in these \emph{0} values.
\item
  Contrary to my initial assumption (people who are more active should
  sleep for longer duration), I observed that there is \emph{no
  correlation} between increased distance tracking and sleeping for
  longer duration.
\item
  Final results show that people who were moderately active diring the
  day i.e. people who tracked a distance between 4 Km to 8 Km had an
  average sleep of 6 hours to 8 hours and beyond.
\end{itemize}

\hypertarget{recommendation}{%
\subsection{Recommendation}\label{recommendation}}

After analyzing the fitbit smart device in order to have its comparision
with Bellabeat's Leaf, I can conclude with the following recommedations-

\begin{itemize}
\item
  Focus on advertisements to market the theory that tracking increased
  distances using Leaf device may result in better/relaxed sleep.
\item
  We should also consider why there were 5\% missing values in Fitbit
  devices and avoid having those similar trends in Leaf by focusing on R
  \& D as well.
\end{itemize}

** Increasing battery capacity to avoid discharge during the nights or
decrease battery usage through \emph{low power mode} etc.

\end{document}
